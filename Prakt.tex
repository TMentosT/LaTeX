\documentclass[10pt]{article}
\usepackage{Prakt-oform}

\begin{document}
\title
    {Образец статьи  \\<<Компьютерные науки и информационные технологии>>}
\author
    {Автор~И.\,О.\authorrefmark{1}, Соавтор~И.\,О.\authorrefmark{2}}
\email
    {\authorrefmark{1}author1@site.ru, \authorrefmark{2}author2@site.ru}
\organization
    {\authorrefmark{1}Организация, Город, Страна; \authorrefmark{2}}


\abstract
    {Данный текст является образцом оформления статьи.
    Аннотация кратко характеризует основную цель работы,
    особенности предлагаемого подхода и~основные результаты.}
\keywords
    {образец, пример, оформление}

\maketitle

\section*{Введение}
После аннотации, но перед первым разделом,
может идти неформальное введение,
описание предметной области,
обоснование актуальности задачи,
краткий обзор известных результатов,
и~т.\,п. В любом случае, структура статьи остается прерогативой авторов.

\section{Название раздела}
Данный документ демонстрирует оформление статьи,
подаваемой на международную конференцию.

\paragraph{Название параграфа.}

Нет никаких ограничений на~количество разделов и~параграфов в~статье.

\paragraph{Теоретическую часть работы}(если таковая имеется) желательно структурировать
с~помощью окружений
Def, Axiom, Hypothesis, Problem, Lemma, Theorem, Corollary, State, Example, Remark.

\begin{Def}
    Математический текст \emph{хорошо структурирован},
    если в~нём выделены определения, теоремы, утверждения, примеры, и~т.\,д.,
    а~неформальные рассуждения (мотивации, интерпретации)
    вынесены в~отдельные параграфы.
\end{Def}

\begin{State}
    Мотивации и~интерпретации наиболее важны для понимания сути работы.
\end{State}

\begin{Theorem}
    Не~менее $90\%$ коллег, заинтересовавшихся Вашей статьёй,
    прочитают в~ней не~более~$10\%$ текста,
    причём это будут именно те~разделы, которые не содержат формул.
\end{Theorem}

\begin{Remark}
    Выше показано применение окружений
    Def, Theorem, State, Remark.
\end{Remark}

\section{Некоторые формулы}

Образец формулы: $f(x_i,\alpha^\gamma)$.

Образец выключной формулы без номера:
\[
    y(x,\alpha) =
    \begin{cases}
        -1, & \text{если } f(x,\alpha)<0;  \\
        +1, & \text{если } f(x,\alpha)\geq 0.
    \end{cases}
\]

Образец выключной формулы с номером:
\begin{equation}\label{eq:cases01}
    F(\mathbf{p}) \to \min ,\quad
    F(\mathbf{p}) =
        \begin{cases}
            f(\mathbf{p},\mathbf{s}_0), & \mathbf{p} \in \Omega _p^{(st)}(\mathbf{s}_0), \\
            + \infty , & \mathbf{p} \notin \Omega _p^{(st)}(\mathbf{s}_0),
        \end{cases}
\end{equation}

Образец выключной формулы, разбитой на две строки с~помощью окружения multline:
\begin{multline}\label{eq:multline01}
\psi (x,y,t)=
    \frac{(t-t_{2})(t-t_{3})(t-t_{4})}{(t_{1}-t_{2})(t_{1}-t_{3})(t_{1}-t_{4})}  f_{1}(x,y) +
    \frac{(t-t_{1})(t-t_{3})(t-t_{4})}{(t_{2}-t_{1})(t_{2}-t_{3})(t_{2}-t_{4})}  f_{2}(x,y) + \\
   +\frac{(t-t_{1})(t-t_{2})(t-t_{4})}{(t_{3}-t_{1})(t_{3}-t_{2})(t_{3}-t_{4})}  f_{3}(x,y) +
    \frac{(t-t_{1})(t-t_{2})(t-t_{3})}{(t_{4}-t_{1})(t_{4}-t_{2})(t_{4}-t_{3})}  f_{4} (x,y).
\end{multline}

Образец набора нумерованных формул, выровненных с помощью окружения align:
\begin{align}
	\vartheta_{H1}(t) ={}& \left(0.85 \frac{\rho_1(t)}{\rho_0}\right)^{k - 1} \vartheta_0, \label{eq:al1}\\
	\vartheta_{H2}(t) ={}& \left(0.85 \frac{\rho_2(t)}{\rho_1}\right)^{k - 1} \vartheta_{H1}, \label{eq:al2}\\
	\vartheta_{H3}(t) ={}& \left(0.85 \frac{\rho_3(t)}{\rho_2}\right)^{k - 1} \vartheta_{H2}. \label{eq:al3}
\end{align}

Образец набора формул под одним номером, выровненных с помощью окружения gathered
\begin{equation}\label{eq:gather1}
    \begin{gathered}
        (1 + \gamma \lambda )u''''(x,\lambda ) +
        {a_x}[({m_2} + 1 - x)u'(x,\lambda )]' +
        {\lambda ^2}u(x,\lambda ) =  - \delta _j^1 - \delta _j^3x, \\
        u(0,\lambda ) = 0;\quad
        u'(0,\lambda ) = 0;\quad
        u(1,\lambda ) = \delta _j^2;\quad
        u'(1,\lambda ) = \delta _j^4;\quad
        j = 1,2,3,4.
    \end{gathered}
\end{equation}

Образец сложного многострочного набора формул под одним номером
\begin{equation}\label{eq:complex1}
\begin{gathered}
J_0\ddot\beta_0 = - p_1\dot\beta_0 - p_2\beta _0 + \mathbf{S}(\beta _1 + \beta_2),\quad
   m_1\ddot y_1 = (1 + m_1 + m_2)\beta_0 + P_1 - F_e,\\
J_0\ddot\beta_0 + J_1\ddot\beta_1 = M_1,\quad
  m_2[(1 + a)\ddot\beta_1 + \ddot y_1 + \ddot y_2] = P_2 + a_x m_2\beta_2,\\
J_2(\ddot\beta_1 + \ddot\beta_2) = M_2 - a P_2,\quad
  \mathbf{S}(.) = p_3d()/dt + p_4 \cdot () + p_5\int\limits_0^t ()\,dt, \\
\ddot u + u'''' + \gamma \dot u'''' + a_x[(m_2 + (1 - x))u']' =  - \ddot y_1 - x\ddot\beta_1,\quad
  ()' = \partial ()/\partial x,\\
u(0,t) = 0;\quad
u'(0,t) = 0,\quad
u(1,t) = y_2(t),\; u'(1,t) = \beta_2(t),\\
M_1 = u''(0,t) + \gamma \dot u''(0,t), \quad
P_1 = - u'''(0,t) - \gamma \dot u'''(0,t),\\
M_2 = - u''(1,t) - \gamma \dot u''(1,t),\quad
  P_2 = u'''(0,t) + \gamma \dot u'''(0,t),\\
\begin{split}
\beta_0(0) = \beta_1(0) = \beta_2(0) = \dot\beta_0(0) = \dot\beta_1(0) ={}& \dot\beta_2(0) = y_1(0) = \\
        {} = y_2(0) = \dot y_1(0) & {} = \dot y_2(0) = 0,\quad u(x,0) = \dot u(x,0) = 0.
\end{split}
\end{gathered}
\end{equation}

Образцы ссылок: формулы~\eqref{eq:cases01}, \eqref{eq:multline01} и система~\eqref{eq:complex1}.

\section{Таблицы}
Пример таблицы.
\begin{table}[!ht]
\caption{Нумерованная таблица}\label{tab:table1}
\centering\small
\begin{tabular}{|c|c|c|c|c|c|} \hline
$t$  & Шаг $k$ & \parbox[c][2.5em]{5em}{\centering Прогноз $Y_{\text{р}}(N+k)$} & $U(k)$ & \parbox{9em}{\centering Нижняя граница $Y_{\text{р}}(N+k)-U(k)$} & \parbox{9em}{\centering Верхняя граница $Y_{\text{р}} (N+k)+U(k)$} \cr \hline
10 & 1 & 90.3613 & 2.0310 & 88.3303 & 92.3923 \cr \hline
11 & 2 & 92.6784 & 2.1494 & 90.5290 & 94.8278 \cr \hline
12 & 3 & 94.9954 & 2.2814 & 92.7140 & 97.2768 \cr \hline
13 & 4 & 97.3125 & 2.4248 & 94.8877 & 99.7373 \cr \hline
14 & 5 & 99.6296 & 2.5777 & 97.0518 & 102.2073 \cr \hline
\end{tabular}
\end{table}

Еще один пример: таблица без номера (допускается только в случае, когда в статье только одна таблица).
\begin{table}[!ht]
\centering \small
\ncaption{Таблица без номера} % замена команде \caption
\begin{tabular}{|p{1.6in}|p{3.1in}|} \hline
\centering{Стадии} & \centering{Результат} \cr \hline
1. Обоснование создания АС  & Научно"=технический отчет \cr \hline
2. Техническое задание & Техническое задание  \cr \hline
3. Технический проект & Документы спецификаций вариантов использования, модель данных и БД, модель пользовательского интерфейса, сценарии тестов \cr \hline
4. Рабочая документация & Комплект пользовательской документации АИС  \cr \hline
5. Ввод в действие & Готовая АИС \cr \hline
\end{tabular}
\end{table}
\section*{Заключение}
Если этот раздел присутствует, то он не~должен дословно повторять аннотацию.
Обычно здесь отмечают,
каких результатов удалось добиться,
какие проблемы остались открытыми.

\begin{thebibliography}{1}
\bibitem{author09anyscience}
    \BibAuthor{Author\;N.}
    \BibTitle{Paper title}~//
    10-th Int'l. Conf. on Anyscience, 2009.~--- Vol.\,11, No.\,1.~--- Pp.\,111--122.
\bibitem{myHandbook}
    \BibAuthor{Автор\;И.\,О.}
    Название книги.~---
    Город: Издательство, 2009.~--- 314~с.
\bibitem{author09first-word-of-the-title}
    \BibAuthor{Автор\;И.\,О.}
    \BibTitle{Название статьи}~//
    Название конференции или сборника,
    Город:~Изд-во, 2009.~--- С.\,5--6.
\bibitem{author-and-co2007}
    \BibAuthor{Автор\;И.\,О., Соавтор\;И.\,О.}
    \BibTitle{Название статьи}~//
    Название журнала.~--- 2007.~--- Т.\,38, \No\,5.~--- С.\,54--62.
\bibitem{bibUsefulUrl}
    \BibUrl{www.site.ru}~---
    Название сайта~--- 2007.
\bibitem{voron06latex}
    \BibAuthor{Воронцов~К.\,В.}
    \LaTeXe\ в~примерах.~---
    2006.~---
    \BibUrl{http://www.ccas.ru/voron/latex.html}.
\bibitem{Lvovsky03}
    \BibAuthor{Львовский~С.\,М.} Набор и вёрстка в пакете~\LaTeX.~---
    3-е издание.~---
    Москва:~МЦHМО, 2003.~--- 448~с.
\end{thebibliography}

\end{document}
