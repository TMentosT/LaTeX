\documentclass[a4paper,12pt,twoside]{article}
\usepackage{FEMJ_2015_2}
\usepackage{amsbib}

\newtheorem{defi}{\indent Определение}
%\newtheorem{thm}{\indent  Теорема}
%\newtheorem{lem}{\indent  Лемма}
\newtheorem{cor}{\indent  Следствие}
\newtheorem{sta}{\indent  Утвержнение}

%\renewcommand{\thedefi}{\arabic{part}\arabic{defi}}
%\renewcommand{\thethm}{\arabic{thm}}
%\renewcommand{\thecor}{\arabic{cor}}
%\renewcommand{\thelem}{\arabic{lem}}
%\renewcommand{\thesta}{\arabic{part}\arabic{sta}}
%\renewcommand{\theequation}{\arabic{equation}}

%\renewcommand{\baselinestretch}{1,5}

\begin{document}
\count0=1\sbox{\pagebox}{1--6}                       %%%%%%%%%%%%%%%% Исправить

\UDC{517.5}
\AMS{30A10, 30C10, 30C15}

\SupportedBy{Работа выполнена при финансовой поддержке Российского фонда
фундаментальных исследований (грант 11-01-00038)}

\Submitted{12 марта 2012 г.}% дата получения работы

\Title{Неравенства для модулей рациональных функций}

\Author{С.\,И.~Калмыков}{Дальневосточный федеральный университет, 690950, г.~Владивосток, ул.~Суханова,~8;
                         Институт прикладной математики ДВО РАН, 690041, Владивосток, ул. Радио, 7}
{sergeykalmykov@inbox.ru}

\markboth{Неравенства для модулей рациональных функций}{С.\,И.~Калмыков}      %%%%%%%%%%%% Левый и правый Колонтитулы

\Summary{Kalmykov~S.\,I.}{Inequalities for Modulus of Rational Functions}  % {ФИО на английском} {аннотация на английском}
{Inequalities for modulus of rational functions
with prescribed poles lying in the exterior of the unit disk were
obtained in this research. The case when the rational function has
no zeros in the unit disk has also been considered.

\keywords{inequalities for rational functions, Blaschke product, Schwarz lemma.}}  % {ключевые слова на английском}}

\makeface

\Abstract{В работе получены неравенства для модулей
рациональных функций с предписанными полюсами, лежащими во
внешности единичного круга. Рассмотрен также случай, когда
рациональная функция не имеет нулей в единичном круге.

Ключевые слова: \tit{неравенства для рациональных функций, произведение Бляшке, лемма Шварца.}}


\section*{Введение}

Введем обозначения
$$
{\cal P}_n^c := \left\{p(z): p(z) = \sum\limits_{k=0}^{n}c_kz^k,
\,\, c_k \in \mathbb{C}, \, c_n\ne 0\right\},
$$
$$
{\cal R}_{n,m}^c := \left\{r(z): r(z) =
\frac{p(z)}{\prod\limits_{k=1}^{n}(z-a_k)}, \, p(z) \in {\cal
P}_m^c, a_k \in \mathbb{C}, \ |a_k|>1, k=\overline{1,n}\right\}.
$$

Для функции $r(z)\in {\cal R}_{n,m}^c$ определим функцию
$$
B(z) = \prod\limits_{k=1}^{n}\frac{1-\overline{a}_kz}{z-a_k}
$$
и положим
$$
||f||:= \max\limits_{|z|=1}|f(z)|,
$$
$$
r^{*}(z)=B(z) \overline{r\left(\frac{1}{\overline{z}}\right)},
$$
а для полинома $p(z)\in {\cal P}_n^c$ рассмотрим ``взаимный''
полином \cite[c. 256]{Gov}
$$
p^{*}(z)=z^n \overline{p\left(\frac{1}{\overline{z}}\right)}.
$$

Оценкам роста полиномов и рациональных функций посвящено большое
количество работ (см., например, статьи \cite{Ank}--\cite{ShLi},
а также монографию \cite[глава 12]{Rah02} и библиографию в них).
Классическим является неравенство


$$
\max\limits_{|z|=\rho\ge1}|p(z)|\leq \rho^n||p||,
$$
справедливое для всех полиномов класса ${\cal P}_n^c$. Если также
известно, что у полинома указанного класса нет нулей в единичном
круге, то справедливо неравенство Анкени~--- Ривлина (см.
\cite{Ank})

$$
\max\limits_{|z|=\rho\ge1}|p(z)|\leq \frac{\rho^n+1}{2}||p||.
$$
Усиление этого неравенства было получено в работе \cite{Dub01}.

Для функций $r(z)\in {\cal R}_{n,m}^c$ справедливо неравенство
(см. \cite{Gov}, \cite{Gon})
\begin{equation}\label{rat} |r(z)|\leq |z|^{(m-n)_{+}} |B(z)|\,||r||, \, \, \,
|z|>1,
\end{equation}
где $x_{+}=\max(x,0)$. Равенство в $(\ref{rat})$ достигается при
$r(z)=z^kB(z), \, k\in \mathbb{N}_0$. Если у рациональной функции
класса ${\cal R}_{n,m}^c$, $m\leq n$,  нет нулей в единичном
круге, то верно  неравенство
\begin{equation}\label{ranri}
 |r(z)|\leq \frac{|B(z)|+1}{2}||r||, \, \,
\, |z|>1,
\end{equation}
с равенством для функции $r(z)=\alpha B(z)+\beta$, где
$|\alpha|=|\beta|$ (см. \cite{Gov}).

В недавней работе автора \cite{Kalm} были получены точные
двусторонние неравенства при дополнительном ограничении на $|z|$,
дополняющие и уточняющие неравенство (\ref{rat}). Цель настоящей
заметки -- усиление и дополнение результатов работ \cite{Gov} и
\cite{Kalm} без дополнительных ограничений на $|z|$. Нам
понадобится следующая лемма 1, вытекающая из леммы Шварца (см.
\cite[стр. 320]{Gol}). Другие приложения этого результата можно
найти, например, в работах \cite{Dub01}, \cite{GovRah}.

\begin{lem} Пусть $f(z)$~--- аналитическая в единичном круге $|z|<1$ функция
такая, что $|f(z)|< 1$ при $|z|<1$. Тогда при $|z|<1$
\begin{equation}\label{ine}
|f(z)|\leq\frac{|z|+|f(0)|}{1+|f(0)||z|},
\end{equation}
причем при $z\ne0$ равенство в $(\ref{ine})$ будет достигаться
только в том случае, когда $f(z)=\varepsilon
\dfrac{z+a}{1+\overline{a}z}$, $|\varepsilon|=1$, $|a|<1$, и $\arg
z=\arg a$, если $a\ne 0$.
\end{lem}

\section*{Основные результаты}

\begin{thm} Пусть рациональная функция $r(z)$ принадлежит классу ${\cal
R}_{n,m}^c$ и $||r||=1$. Тогда справедливо неравенство
\begin{equation}\label{rin}
 |r(z)|\leq \frac{\displaystyle
\prod\limits_{k=1}^{n}|a_k|+|c_mz|}{\displaystyle
|c_m|+|z|\prod\limits_{k=1}^{n}|a_k|}|B(z)||z|^{m-n}, \, \, \, |z|>1.
\end{equation}


Если $r(z)=z^kB(z), \, k\in \mathbb{N}_0$, то неравенство
$(\ref{rin})$ становится равенством для любого $z$, $|z|>1$.
\end{thm}
\medskip

\begin{proof} Рассмотрим функцию
$$
f(\zeta)= \zeta^{m-n}
\overline{r\left(\frac{1}{\overline{\zeta}}\right)}B(\zeta)=
\frac{\overline{c_m}}{(-1)^n\prod\limits_{k=1}^{n}a_k}+ ... .
$$

Она аналитическая в единичном круге, так как полюса функции
$\overline{r(1/\overline{\zeta})}$ являются нулями функции
$\zeta^{m-n}B(\zeta)$ с учетом кратности. Тогда из того, что
$$
|f(\zeta)|= |\zeta^{m-n}|
\left|\overline{r\left(\frac{1}{\overline{\zeta}}\right)}\right||B(\zeta)|=\left|\overline{r\left(\frac{1}{\overline{\zeta}}\right)}\right| \leq 1, \, \, \, |z|=1,
$$
следует, что $f(z)$
 ограничена по модулю единицей в единичном круге и, таким образом, удовлетворяет условиям леммы 1, при этом
$f(0)=(-1)^n\overline{c_m}/\left(\prod\limits_{k=1}^{n}a_k\right)$.

Выпишем неравенство (\ref{ine}) для функции $f(\zeta)$
$$
|\zeta^{m-n}
\overline{r\left(\frac{1}{\overline{\zeta}}\right)}B(\zeta)|\leq
\frac{|\zeta|+|c_m|/\left(\prod\limits_{k=1}^{n}|a_k|\right)}{1+|\zeta||c_m|/\left(\prod\limits_{k=1}^{n}|a_k|\right)}, \, \, \, \, |\zeta|<1.
$$


Делая замену переменной $z=1/\overline{\zeta}$, приходим к
неравенству
$$
|\overline{r(z)}B(1/\overline{z})|\leq
|z^{m-n}|\frac{1+|z||c_m|/\left(\prod\limits_{k=1}^{n}|a_k|\right)}{|z|+|c_m|/\left(\prod\limits_{k=1}^{n}|a_k|\right)}, \, \, \, \, |z|>1,
$$
а замечая, что $\overline{B(1/\overline{z})}=1/B(z)$, получаем
неравенство (\ref{rin}).

 Пусть теперь $r(z)=z^kB(z), \, k\in
\mathbb{N}_0$, тогда ясно, что $m-n=k$, а
$|c_m|=\prod\limits_{k=1}^{n}|a_k|$, и равенство в (\ref{rin})
очевидно. Теорема доказана.
\end{proof}

Из принципа максимума модуля и убывания при $x\geq1$ функции
$h(x)=\dfrac{1+ax}{a+x}, \, 0<a<1$, получаем, что
$$
\frac{\displaystyle
\prod\limits_{k=1}^{n}|a_k|+|c_mz|}{\displaystyle
|c_m|+|z|\prod\limits_{k=1}^{n}|a_k|}\leq1, \, \, \, \mbox{для всех $z$ таких, что} \, \, |z|>1.
$$
Таким образом, неравенство (\ref{rin}) улучшает неравенство
(\ref{rat}).


\begin{cor} Пусть полином $p(z)$ принадлежит классу ${\cal P}_n^c$ и
$||p||~=~1$, тогда  выполняется неравенство
$$
|p(z)|\leq \frac{\displaystyle
1+|c_nz|}{\displaystyle
|c_n|+|z|}|z|^{n}, \, \, \, |z|>1.
$$

Равенство для любой точки $z: |z|>1$  достигается в случае
полинома $P(z) = c_nz^n$, $ |c_n|=1$. \end{cor}

\begin{cor}  Пусть рациональная функция $r(z)$ принадлежит
классу ${\cal R}_{n,m}^c$ и $||r||=1$. Тогда для $|z|>1$
справедливо неравенство
$$
|r^{*}(z)|\leq \frac{\displaystyle
\prod\limits_{k=1}^{n}|a_k|+|c_0z|}{\displaystyle
|c_0|+|z|\prod\limits_{k=1}^{n}|a_k|}|B(z)|.
$$
\end{cor}

\begin{proof} Из того, что $r(z)\in {\cal R}_{n,m}^c$,
следует
$$
r(z) =
\frac{p(z)}{\prod\limits_{k=1}^{n}(z-a_k)}.
$$
Тогда
$$
\frac{r^{*}(z)}{z^{n-m}}=z^{m-n}B(z)\overline{r\left(\frac{1}{\overline{z}}
\right)}=\frac{p^{*}(z)}{\prod\limits_{k=1}^{n}(z-a_k)},
$$
то есть функция $r^{*}(z)/z^{n-m}$  принадлежит классу ${\cal
R}_{n,m}^c$. Так как $||r^{*}||=||r||=1$, то к $r^{*}(z)/z^{n-m}$
применима теорема 1. Следствие доказано.
\end{proof}

\begin{lem} Пусть $r(z)\in {\cal R}_{n,m}^c$. Если  функция $r(z)$ не
имеет нулей в единичном круге $|z|<1$, то
\begin{equation}\label{rr}
|z^{n-m}r(z)|\leq|r^{*}(z)|, \,\,\, |z|>1.
\end{equation}

Для функции $r(z)=\alpha B(z)+\beta$, где $|\alpha|=|\beta|$,
справедливо равенство в $(\ref{rr})$.
\end{lem}

\begin{proof} Так как функция $r$ не имеет нулей в
единичном круге, то рациональная функция
$$
\frac{r^{*}(z)}{z^{n-m}}=z^{m-n}B(z)\overline{r\left(\frac{1}{\overline{z}}
\right)}=\frac{p^{*}(z)}{\prod\limits_{k=1}^{n}(z-a_k)}
$$
имеет конечные нули только в замкнутом единичном круге.
Следовательно, функция \\ $z^{n-m}r(z)/r^{*}(z)$ аналитическая в
$|z|\geq1$. На единичной окружности $|z|=1$
$$
\left|\frac{z^{n-m}r(z)}{r^{*}(z)}\right|=\left|\frac{p(z)}{p^{*}(z)}\right|=1.
$$
Из принципа максимума модуля следует
$$
\frac{|z^{n-m}r(z)|}{|r^{*}(z)|}\leq1, \, \, \, \, |z|\geq1,
$$
что эквивалентно  неравенству (\ref{rr}). Случай равенства
проверяется непосредственно. Лемма доказана.
\end{proof}

Из следствия 2 и леммы 2 вытекает теорема.

\begin{thm} Пусть рациональная функция $r(z)$ принадлежит классу ${\cal
R}_{n,m}^c$, не имеет нулей в отрытом единичном круге, и
$||r||=1$. Тогда справедливо неравенство
\begin{equation}\label{rinn}
 |r(z)|\leq \frac{\displaystyle
\prod\limits_{k=1}^{n}|a_k|+|c_0z|}{\displaystyle
|c_0|+|z|\prod\limits_{k=1}^{n}|a_k|}|B(z)||z|^{m-n}, \, \, \, |z|>1.
\end{equation}
\end{thm}


Неравенство $(\ref{rinn})$ в условиях предыдущего следствия, как и
неравенство $(\ref{rin})$, улучшает неравенство $(\ref{rat})$.
Следующая теорема уточняет неравенство $(\ref{ranri})$.


\begin{thm} Пусть рациональная функция $r(z)$, не имеющая нулей в
отрытом единичном круге, принадлежит классу ${\cal R}_{n,m}^c$,
$m\leq n$. Тогда для $z: |z|>1$ справедливо неравенство
$$\
|r(z)|\leq \frac{|B(z)|+1}{|z|^{n-m}+1}||r||.
$$

Равенство достигается для функции $r(z)=\alpha B(z)+\beta$, где
$|\alpha|=|\beta|$. \end{thm}

{\it Доказательство} немедленно следует из леммы 2 и следующего
утверждения.

 \begin{lem} {\rm \cite{Gov}}
 Пусть рациональная функция $r(z)$ принадлежит классу ${\cal
R}_{n,m}^c$, $m\leq n$ и $||r||=1$. Тогда для $z: |z|>1$
справедливо неравенство
$$
|r(z)|+|r^{*}(z)|\leq\left(|B(z)|+1\right).
$$
\end{lem}


%%====================================
 \begin{thebibliography}{12}
 \setlength{\parsep}{0pt}\setlength{\itemsep}{3pt}
%%====================================

\Bibitem{Ank}
\by N.\,C.~Ankeny, T.\,J.~Rivlin
\paper On a theorem of S.~Bernstein
\jour Pacific J. Math.
\yr 1955
\vol 5
\issue suppl. 2
\pages 849--852

\RBibitem{Gonch2}
\by А.\,А.~Гончар
\paper Оценки роста рациональных функций и некоторые их приложения
\jour Матем. сб.
\yr 1967
\vol 72(114)
\issue 3
\pages
489--503

\Bibitem{Gov}
\by N.\,K.~Govil, R.\,N.~Mohapatra
\paper Inequalities for Maximum Modulus of Ratioanl Functions with Prescribed Poles
\inbook Approximation Theory: In Memory of A.\,K.~Varma
\publ Marcel Dekker, Inc.
\publaddr New York
\yr 1998
\pages 255--263

\Bibitem{Qaz92}
\by M.\,A.~Qazi
\paper On the maximum modulus of polynomials
\jour Proc. Amer. Math. Soc.
\yr 1992 \vol 115
\issue 2
\pages 337--349

\Bibitem{ShLi}
\by W.\,M.~Shah, A.~Liman
\paper Integral estimates for the family of B-operators
\jour Operators and matrices
\yr 2011
\vol 5
\issue 1
\pages 79--87

\Bibitem{Rah02}
\by Q.\,I.~Rahman,  G.~Schmeisser
\book Analytic theory of polynomials
\publaddr Oxford
\publ Oxford University Press
\yr 2002

\RBibitem{Dub01}
\by В.\,Н.~Дубинин
\paper О применении леммы Шварца к неравенствам для целых функций с ограничениями на нули
\jour Зап. научн. сем. ПОМИ
\yr 2006
\vol 337
\pages 101--112

\RBibitem{Gon}
\by А.\,А.~Гончар
\paper О задачах Е.\,И.~Золотарева, связанных с рациональными функциями
\jour Матем. сб.
\yr 1969
\vol 78(120)
\issue 4
\pages 640--654

\RBibitem{Kalm}
\by С.\,И.~Калмыков
\paper Об оценке модуля рациональной функции
\jour Зап. научн. сем. ПОМИ
\yr 2009
\vol 371
\pages 109--116

\RBibitem{Gol}
\by Г.\,М.~Голузин
\book Геометрическая теория функций комплексного переменного
\publaddr М
\publ Наука
\yr 1966

\Bibitem{GovRah}
\by N.\,K.~Govil, Q.\,I.~Rahman, G.~Schmeisser
\paper On the derivative of a polynomial
\jour Illinois Journal of Mathematics
\yr 1979
\vol 23
\pages 319--329

\end{thebibliography}

\EndArticle

\end{document}